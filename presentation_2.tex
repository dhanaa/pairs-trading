%%%%%%%%%%%%%%%%%%%%%%%%%%%%%%%%%%%%%%%%%
% KOMA-Script Presentation
% LaTeX Template
% Version 1.0 (3/3/13)
%
% This template has been downloaded from:
% http://www.LaTeXTemplates.com
%
% Original Authors:
% Marius Hofert (marius.hofert@math.ethz.ch)
% Markus Kohm (komascript@gmx.info)
% Described in the PracTeX Journal, 2010, No. 2
%
% License:
% CC BY-NC-SA 3.0 (http://creativecommons.org/licenses/by-nc-sa/3.0/)
%
%%%%%%%%%%%%%%%%%%%%%%%%%%%%%%%%%%%%%%%%%

%----------------------------------------------------------------------------------------
%	PACKAGES AND OTHER DOCUMENT CONFIGURATIONS
%----------------------------------------------------------------------------------------

\documentclass[
paper=128mm:96mm, % The same paper size as used in the beamer class
fontsize=11pt, % Font size
pagesize, % Write page size to dvi or pdf
parskip=half-, % Paragraphs separated by half a line
]{scrartcl} % KOMA script (article)

\linespread{1.12} % Increase line spacing for readability

%------------------------------------------------
% Colors
\usepackage{xcolor}	 % Required for custom colors
% Define a few colors for making text stand out within the presentation
\definecolor{mygreen}{RGB}{44,85,17}
\definecolor{myblue}{RGB}{34,31,217}
\definecolor{mybrown}{RGB}{194,164,113}
\definecolor{myred}{RGB}{255,66,56}
% Use these colors within the presentation by enclosing text in the commands below
\newcommand*{\mygreen}[1]{\textcolor{mygreen}{#1}}
\newcommand*{\myblue}[1]{\textcolor{myblue}{#1}}
\newcommand*{\mybrown}[1]{\textcolor{mybrown}{#1}}
\newcommand*{\myred}[1]{\textcolor{myred}{#1}}
%------------------------------------------------

%------------------------------------------------
% Margins
\usepackage[ % Page margins settings
includeheadfoot,
top=3.5mm,
bottom=3.5mm,
left=5.5mm,
right=5.5mm,
headsep=6.5mm,
footskip=8.5mm
]{geometry}
%------------------------------------------------

%------------------------------------------------
% Fonts
\usepackage[T1]{fontenc}	 % For correct hyphenation and T1 encoding
\usepackage{lmodern} % Default font: latin modern font
%\usepackage{fourier} % Alternative font: utopia
%\usepackage{charter} % Alternative font: low-resolution roman font
\renewcommand{\familydefault}{\sfdefault} % Sans serif - this may need to be commented to see the alternative fonts
%------------------------------------------------

%------------------------------------------------
% Various required packages
\usepackage{amsthm} % Required for theorem environments
\usepackage{bm} % Required for bold math symbols (used in the footer of the slides)
\usepackage{graphicx} % Required for including images in figures
\usepackage{tikz} % Required for colored boxes
\usepackage{booktabs} % Required for horizontal rules in tables
\usepackage{multicol} % Required for creating multiple columns in slides
\usepackage{lastpage} % For printing the total number of pages at the bottom of each slide
\usepackage[english]{babel} % Document language - required for customizing section titles
\usepackage{microtype} % Better typography
\usepackage{tocstyle} % Required for customizing the table of contents
%------------------------------------------------

%------------------------------------------------
% Slide layout configuration
\usepackage{scrpage2} % Required for customization of the header and footer
\pagestyle{scrheadings} % Activates the pagestyle from scrpage2 for custom headers and footers
\clearscrheadfoot % Remove the default header and footer
\setkomafont{pageheadfoot}{\normalfont\color{black}\sffamily} % Font settings for the header and footer

\usepackage{hyperref}

% Sets vertical centering of slide contents with increased space between paragraphs/lists
\makeatletter
\renewcommand*{\@textbottom}{\vskip \z@ \@plus 1fil}
\newcommand*{\@texttop}{\vskip \z@ \@plus .5fil}
\addtolength{\parskip}{\z@\@plus .25fil}
\makeatother

% Remove page numbers and the dots leading to them from the outline slide
\makeatletter
\newtocstyle[noonewithdot]{nodotnopagenumber}{\settocfeature{pagenumberbox}{\@gobble}}
\makeatother
\usetocstyle{nodotnopagenumber}

\AtBeginDocument{\renewcaptionname{english}{\contentsname}{\Large Outline}} % Change the name of the table of contents
%------------------------------------------------

%------------------------------------------------
% Header configuration - if you don't want a header remove this block
\ihead{
\hspace{-2mm}
\begin{tikzpicture}[remember picture,overlay]
\node [xshift=\paperwidth/2,yshift=-\headheight] (mybar) at (current page.north west)[rectangle,fill,inner sep=0pt,minimum width=\paperwidth,minimum height=2\headheight,top color=mygreen!64,bottom color=mygreen]{}; % Colored bar
\node[below of=mybar,yshift=3.3mm,rectangle,shade,inner sep=0pt,minimum width=128mm,minimum height =1.5mm,top color=black!50,bottom color=white]{}; % Shadow under the colored bar
shadow
\end{tikzpicture}
\color{white}\runninghead} % Header text defined by the \runninghead command below and colored white for contrast
%------------------------------------------------

%------------------------------------------------
% Footer configuration
\newlength{\footheight}
\setlength{\footheight}{8mm} % Height of the footer
\addtokomafont{pagefoot}{\footnotesize} % Small font size for the footnote

\ifoot{% Left side
\hspace{-2mm}
\begin{tikzpicture}[remember picture,overlay]
\node [xshift=\paperwidth/2,yshift=\footheight] at (current page.south west)[rectangle,fill,inner sep=0pt,minimum width=\paperwidth,minimum height=3pt,top color=mygreen,bottom color=mygreen]{}; % Green bar
\end{tikzpicture}
\myauthor\ \raisebox{0.2mm}{$\bm{\vert}$}\ \myuni % Left side text
}

\ofoot[\pagemark/\pageref{LastPage}\hspace{-2mm}]{\pagemark/\pageref{LastPage}\hspace{-2mm}} % Right side
%------------------------------------------------

%------------------------------------------------
% Section spacing - deeper section titles are given less space due to lesser importance
\usepackage{titlesec} % Required for customizing section spacing
\titlespacing{\section}{0mm}{0mm}{0mm} % Lengths are: left, before, after
\titlespacing{\subsection}{0mm}{0mm}{-1mm} % Lengths are: left, before, after
\titlespacing{\subsubsection}{0mm}{0mm}{-2mm} % Lengths are: left, before, after
\setcounter{secnumdepth}{0} % How deep sections are numbered, set to no numbering by default - change to 1 for numbering sections, 2 for numbering sections and subsections, etc
%------------------------------------------------

%------------------------------------------------
% Theorem style
\newtheoremstyle{mythmstyle} % Defines a new theorem style used in this template
{0.5em} % Space above
{0.5em} % Space below
{} % Body font
{} % Indent amount
{\sffamily\bfseries} % Head font
{} % Punctuation after head
{\newline} % Space after head
{\thmname{#1}\ \thmnote{(#3)}} % Head spec
	
\theoremstyle{mythmstyle} % Change the default style of the theorem to the one defined above
\newtheorem{theorem}{Theorem}[section] % Label for theorems
\newtheorem{remark}[theorem]{Remark} % Label for remarks
\newtheorem{algorithm}[theorem]{Algorithm} % Label for algorithms
\makeatletter % Correct qed adjustment
%------------------------------------------------

%------------------------------------------------
% The code for the box which can be used to highlight an element of a slide (such as a theorem)
\newcommand*{\mybox}[2]{ % The box takes two arguments: width and content
\par\noindent
\begin{tikzpicture}[mynodestyle/.style={rectangle,draw=mygreen,thick,inner sep=2mm,text justified,top color=white,bottom color=white,above}]\node[mynodestyle,at={(0.5*#1+2mm+0.4pt,0)}]{ % Box formatting
\begin{minipage}[t]{#1}
#2
\end{minipage}
};
\end{tikzpicture}
\par\vspace{-1.3em}}
%------------------------------------------------

%----------------------------------------------------------------------------------------
%	PRESENTATION INFORMATION
%----------------------------------------------------------------------------------------

\newcommand*{\mytitle}{Pairs trading: implementations and experiments of relative value arbitrage strategy} % Title
\newcommand*{\runninghead}{Pairs Trading} % Running head displayed on almost all slides
\newcommand*{\myauthor}{Weiyi Chen} % Presenters name(s)
\newcommand*{\mydate}{\today} % Presentation date
\newcommand*{\myuni}{Tsinghua University --- Yao Class} % University or department

%----------------------------------------------------------------------------------------

\begin{document}

%----------------------------------------------------------------------------------------
%	TITLE SLIDE
%----------------------------------------------------------------------------------------

% Title slide - you may have to tweak a few of the numbers if you wish to make changes to the layout
\thispagestyle{empty} % No slide header and footer
\begin{tikzpicture}[remember picture,overlay] % Background box
\node [xshift=\paperwidth/2,yshift=\paperheight/2] at (current page.south west)[rectangle,fill,inner sep=0pt,minimum width=\paperwidth,minimum height=\paperheight/3,top color=mygreen,bottom color=mygreen]{}; % Change the height of the box, its colors and position on the page here
\end{tikzpicture}
% Text within the box
\begin{flushright}
\vspace{0.6cm}
\color{white}\sffamily
{\bfseries\Large\mytitle\par} % Title
\vspace{0.5cm}
\normalsize
\myauthor\par % Author name
\mydate\par % Date
\vfill
\end{flushright}

\clearpage

%----------------------------------------------------------------------------------------
%	TABLE OF CONTENTS
%----------------------------------------------------------------------------------------

\thispagestyle{empty} % No slide header and footer

\small\tableofcontents % Change the font size and print the table of contents - it may be useful to shrink the font size further if the presentation is full of sections
% To exclude sections/subsections from the table of contents, put an asterisk after \(sub)section like so: \section*{Section Name}

\clearpage

%----------------------------------------------------------------------------------------
%	PRESENTATION SLIDES
%----------------------------------------------------------------------------------------

%------------------------------------------------

\section{1. MFE careers}

\clearpage

%------------------------------------------------

\subsection{Financial jobs}

\begin{itemize}
	\item S\&T: the \textcolor{red}{buying} and \textcolor{red}{selling} of securities
	\item Quant: application of \textcolor{red}{mathematical and statistical methods} to financial and risk management problems
	\item Risk Management: \textcolor{red}{identification, assessment, and prioritization} of risks
	\item Investment Banking: \textcolor{red}{raising capital} by underwriting or acting as the client's agent in the issuance of securities
	\item IT: varied and can cover anything
\end{itemize}

\clearpage

%------------------------------------------------

\begin{itemize}
\item Sell-side trading
	\begin{itemize}
		\item Market making: quotes both a buy and a sell price, make a profit on \textcolor{red}{the bid-offer spread}
		\item Etrading
		\item Quant trading: trading based on \textcolor{red}{quantitative analysis}
	\end{itemize}
\item Buy-side trading
	\begin{itemize}
		\item Algo trading: \textcolor{red}{automated trading}, black-box trading, HFT
		\item Quant trading (difference from sell-side, to identify strategies)
	\end{itemize}
\end{itemize}

\clearpage

%------------------------------------------------

\begin{itemize}
\item Front office / desk quant: implement \textcolor{red}{pricing models} used by traders (money and opportunities to move into trading)
\item Model validation: \textcolor{red}{independently} implement pricing models to check front office models (less stressful)
\item Research: invent new pricing approaches (interesting and learn more)
\item Quant developer:  a \textcolor{red}{programmer} (well-paid and easier to find a job)
\item Statistical arbitrage: find patterns in data to suggest \textcolor{red}{automated} trades (return highly volatile)
\item Capital quant: model the bank's credit exposures and \textcolor{red}{capital} requirements
\end{itemize}

\clearpage

%------------------------------------------------


\subsection{Available jobs for FEer}

\begin{itemize}
\item Risk management
\item Quant or S\&T? (RITC competition video)
\item Somewhat depends on program (Computational Finance, Mathematics Finance, Financial Engineering, Finance)
\item Buy side vs. Sell side
\end{itemize}

\clearpage

%------------------------------------------------


\subsection{How to get an internship}

\begin{itemize}
\item Apply through website or sending emails (Linkedin KCG, Baruch)
\item Set up interviews
\item Final round (super day)
\end{itemize}

\clearpage


%----------------------------------------------------------------------------------------

\begin{itemize}
\item Resume: for the interview (mine)
\item Information, \textbf{Networking is rewarded}
\item Interview skills
\begin{itemize}
	\item Technical (book list)
	\item Behavioral (resume)
	\item \textbf{Personality}
\end{itemize}
\end{itemize}

\clearpage

%------------------------------------------------


\section{2. MFE skills preparation}

\clearpage

%------------------------------------------------

\subsection{Basic preparations}

\begin{itemize}
\item Three fields
	\begin{itemize}
		\item CS programming: \textcolor{red}{C++}, OOP, Python, Matlab, R
		\item Math: \textcolor{red}{calculus, linear algebra, statistics, numerical methods}, ode, pde, stochastic process, stochastic calculus
		\item Finance: \textcolor{red}{financial engineering, economics}, corporate finance, accounting, security analysis, portfolio management
	\end{itemize}
\item Resume \& cover letter (= personal statement)
\item Practice your listening and oral english
\end{itemize}

\clearpage

%------------------------------------------------

\subsection{Advanced preparations}

\begin{itemize}
\item Networking
	\begin{itemize}
		\item with seniors (personal recommendation)
		\item with professors (pre-MFE course, PKU econ, phone or meet)
		\item with practitioners (official recommendation)
	\end{itemize}
\item About research and internship
\item CS skills
\end{itemize}

\clearpage

%------------------------------------------------

\subsection{Recommended books and MOOCs}

\begin{itemize}
\item Required: 
	\begin{itemize}
		\item 150 Most Frequently Asked Questions on Quant Interviews
		\item Coursera: Financial Engineering and Risk Management
		\item My Life as a Quant
	\end{itemize}
\item Elective:
	\begin{itemize}
		\item QuantNet best-selling books
		\item Coursera: Intro to CF and FE, Mathematical methods for QF, \textcolor{red}{computational investing}
		\item WSJ, default page
	\end{itemize}
\end{itemize}

\clearpage

%------------------------------------------------

\section{3. MFE programs insights, personally}

\begin{itemize}
	\item CMU MSCF (good at sales \& trading)
	\item Columbia MFE (best reputation in Asia)
	\item Baruch MFE (high employment statistics)
	\item NYU MathFin (good at quant)
	\item Cornell MFE (good at risk management)
\end{itemize}

\clearpage

%------------------------------------------------

\subsection{rankings \& match}

\begin{itemize}
\item \href{https://www.quantnet.com/mfe-programs-rankings/}{Quantnet ranking}
\item \textcolor{red}{Computation}: Baruch, CMU > Columbia FE > NYU, Cornell, Stanford
\item Finance: Berkeley, CMU, Princeton > Baruch, NYU
\item Math: NYU, Stanford > UCB, Baruch, Cornell
\item Reputation: Princeton, MIT, Columbia> Cornell > CMU, Berkeley, NYU
\item \textcolor{red}{Tuition}: Baruch > Columbia, NYU, Berkeley > CMU, Princeton
\end{itemize}

\clearpage

%------------------------------------------------

\subsection{Q\&A}
Thanks for listening. \\
Contacts: wechat 122782886 or weiyi.alan.chen@gmail.com
\clearpage


\end{document}